\section{Limitations}
\label{sec:discussion}

In this section, a number of limitations for the presented results are discussed; pathways for future research are presented that can lead to more accurate and realistic results.


First, one needs to remember that a ``maximum demand'' scenario is presented, i.e. any trip that can be served by the AMoD service must be served by it. Clearly, this poses an upper bound on the attractiveness of the service. As discussed above, this attractiveness, however, is dependent on a multitude of additional factors. Therefore, in reality, one would expect a lower demand, especially in a transitional phase.


To get closer to the actual demand for an AMoD fleet, where wait times, travel times, monetary costs and other factors are considered by the agents, additional components of the MATSim framework can be used. As demonstrated by \citet{horl_abmtrans17}, MATSim is able to dynamically simulate decision behavior for specific modes of transport. In future research, stated choice surveys and other data methods can be used to develop consistent mode choice models, making it possible to estimate a more realistic, dynamic travel demand for an AMoD fleet through simulation. Also, such a setup will enable further questions about control of fleets, e.g. how to not only minimize wait times, but attract customers overall by varying pricing schemes, operating areas and more.  In the case of Zurich, such surveys are already being conducted \citep{Becker2017} and will be used for the next iteration in this line of research. Furthermore, specifically tailored surveys and simulations will make it possible to explore the potential of a combination of AMoD services with the existing public transport infrastructure \citep{SHEN2018125, WANG2018797, trainPaper}. To allow for dynamic demand decisions, the simulations presented here (with run times of around 1h) need to be executed for a large number of subsequent iterations; setting up and conducting such a study is a challenging task.


Second, fleets in this paper operate ideally; i.e., the vehicles do not need maintenance, recharge or refuel (as has been studied by  \citet{LOEB2018222}, \citet{CHEN2016243}).There are also no restrictions for parking the vehicles and no special dynamics defined for picking up and dropping off passengers. Realistically, the spatial demand is expected to be so high at certain spots in the city that fleet operation - as it has been simulated in this paper - would not be possible due to spatial constraints. In future research, these questions will be addressed, because they are of great practical relevance for implementation of on-demand services in the real world.


Third, the simulations presented assume free-speed travel times in the system. Though the small computed fleet sizes suggest that congestion may become a minor problem, operations could be slowed down to some extent in a more realistic set-up. This would especially affect performance of the rebalancing algorithms, because vehicles would move more slowly and the algorithms would be unable to react to demand as quickly as they did in this study. In fact, because a centrally controlled vehicle fleet would, unlike human drivers, use  route assignments that lead to a congestion-generating self-inhibition of the fleet, it may be necessary to implement intelligent routing algorithms to prevent this phenomenon \citep[e.g.][]{LEVIN2017229, ZHANG201875}. As a prerequisite, detailed calibrations of the scenario network used are currently being performed that will allow for realistic travel times and speeds in future versions of our simulations.


Fourth, we only consider single-occupancy taxi services in this research. With that limitation, we ignore potential benefits that may arise through the intelligent clustering of future and/or diversion of ongoing AMoD trips. Future research will take such operating schemes into account  (e.g. as in \citet{TELLEZ201899}).
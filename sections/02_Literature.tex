\section{Related research}
\label{subs:literatureResearch}

Two-way, station-based, mobility-on-demand systems, e.g. car sharing schemes like ``Mobility'', in Switzerland \citep{katzev2003car} are a well-established part of the modal share of many cities, albeit typically at rather insignificant levels, compared to private and public transportation. These schemes offer flexibility, competitive prices and good service levels. Their popularity is heavily limited by the fact that the vehicle must be dropped off where the journey originated. In contrast, in one-way mobility-on-demand systems, customers can travel with a vehicle (e.g. automated car or bike) from any origin to any destination in the city, which dramatically increases the flexibility of these systems. An assignment model for this kind of service was proposed by \citet{katzev2003car}.


The price for increased flexibility is system imbalance. Due to the spatio-temporal and generally unbalanced characteristics of travel demand, vehicles tend to accumulate at certain locations and be lacking at others. Furthermore, system imbalance is not an exception, but occurs for all demand patterns, except for identical distributions of origins and destinations. This can be seen, for instance, when using queuing-theoretical arguments as shown by \citet{zhang2016control}.


System imbalance leads to drastically decreased service levels and must be countered with targeted repositioning of vehicles from oversupplied to empty areas of the city. This repositioning of vehicles - called rebalancing - makes a substantial contribution to the operational cost of operators; various strategies have been tried to minimize the rebalancing effort. For instance, in bike-sharing schemes, trucks are used to move vehicles from full to empty stations; in \citep{pfrommer2014dynamic}, algorithms are proposed to route these trucks at minimal cost. In \citep{ruch2014rule}, price incentive controllers are proposed to encourage customers to travel to depleted stations at their trips' ends. Rebalancing is also being researched for car sharing schemes; in \citep{smith2013rebalancing}, a scheme is proposed to optimally reposition rebalancing drivers for one-way car sharing schemes. AMoD systems differ mainly in that (i) the vehicles can reposition themselves without the use of transport trucks or auxiliary drivers and (ii) the fleet can be controlled centrally without the agent's decisions interfering. Thus, rebalancing can be carried out effectively and efficiently.


Rebalancing of automated mobility-on-demand systems was first presented as a research  problem by \citet{pavone2011load}. Optimal rebalancing flows for the vehicles were obtained by solving a linear program. In \citep{zhang2016control}, relationship to queuing theoretical concepts is established. In \citep{treleaven2011asymptotically}, the rebalancing effort’s relationship to underlying distributions of origins and destinations is established, showing that for general cases where distributions of origins and destinations are not equal, the total minimal rebalancing distance is strictly more than zero. In \citep{zhang2016model}, the rebalancing problem is solved with a model-predictive control algorithm that performs well, but does not scale to large systems. \citet{MA2017124} propose a rolling-horizon linear programming approach for the optimal dispatching of automated vehicles.


\citet{spieser2014toward} apply two systematic approaches to determine the fleet size of an AMoD system which could fulfill the entire Singapore city travel demand. The analytic formulas presented in \citep{zhang2016control} and \citep{treleaven2011asymptotically} were used to compute both the minimal number of vehicles needed to stabilize open requests, as well as number of vehicles needed to provide an acceptable level of service.  The authors conclude that a fleet size of 25\% of the current vehicle fleet would offer average wait times of around 15 minutes and could reduce external and internal costs of mobility by 50\%.  The study does not compare different fleet control algorithms and does not detail whether congestion effects are taken into account. Its results are remarkable; they are derived using fundamental principles, but are not yet verified in a high accuracy simulation. Furthermore, the study does not highlight potential benefits of \textit{combining} AMoD with conventional public transportation, like metro lines.

% Took this out... I don't find it really convincing and comparable to our study
%In \cite{marczuk2015autonomous} a case study is presented where no private cars can enter the central business district of Singapore. A total of $25,525$ trips is served by autonomous taxis which operate either in station-based or free-floating scheme. In the station-based scheme a set of fixed stations exists where the cars return to after completion of a trip. In the free-floating scheme the cars remain parked at the destination of trips. The simulations are based on the SimMobility agent based simulation platform and include twelve different fleet sizes from $2,000$ to $7,500$ vehicles. The authors conclude that the free-floating scheme can serve $90\%$ of the demand at the maximum fleet size whereas the station-based model can only serve $68 \% $ of the demand. Furthermore they observe mean customer wait times that saturate at approximately $2.5 ~ \textnormal{mins}$ and $6,000$ vehicles. While the station-based and free-floating concepts are compared, a detailed comparison of dispatching and rebalancing operation strategies is not presented. Furthermore results on the fleet efficiency and performance are not shown as well as a more detailed analysis of the wait times (e.g. different wait time quantiles). As the CBD of Singapore attracts much more than $25,525$ trips during a day it would also be interesting to see the results with the full number of trips taking into account routing policies and congestion levels in the city.

% ride-sharing ... rather take the older one
\citet{fagnant2015dynamic} present a case study for Austin, Texas that focuses on the use of shared automated vehicles with ride-sharing capabilities, i.e. vehicles that transport more than one customer under some circumstances. Should only one passenger at a time be able to use the vehicle, they conclude that the same number of trips could be served with 90\% fewer vehicles in a shared automated mobility system than with today's fleet of private conventional cars. The scenario presented for vehicles with a one passenger unit capacity indicated that 10\% of today's vehicle fleet could fulfill the entire demand. % with average wait times of 4.49 min. 
The study does not take the effects of different operational policies on the results into account  and its source-code is not publicly available.

Further studies around the Austin, Taxis case are \citet{review2} and \citet{review1}. The former one introduces congestion patterns to a simulation of automated vehicles that arise from a cell-transmission model. The results show that previous vehicle replacement rates are potentially too optimistic, but no thorough sensitivity analysis on different assumptions for capacity gains is presented. The latter is a first analysis of mode choice behaviour under the availability of automated vehicles. The simulations establish a price analysis for Austin, but under the assumption that trips with wait times longer than 10min do not need to be served. Wait times influencing the passenger choices are sampled at random a priori.

% Reidesharing
\cite{zachariah2014uncongested} present a case study for New Jersey which also focused on the potential of ride-sharing in combination with the local train system. The study concludes that the ride-sharing potential is large, especially during rush-hour and that shared automated vehicles could substantially reduce congestion levels in the city. This study also does not take the effect of different fleet operational policies into account .


In \citep{zhu2017interplay}, the New Jersey demand is served by automated vehicles and a detailed analysis of fleet sizes and resulting empty mileage is performed. A linear program is used to find the optimal distribution of automated vehicles in the early morning for the study area and a heuristic approach, based on Euclidean distance, is used to move vehicles during a simulation day.


In \citep{martinez2017assessing} the authors present a study on the effects of introducing automated taxis and automated shared taxis to the city of Lisbon, Portugal. The agent-based simulation includes 1.2 million trips and three scenarios: a baseline scenario showing the current situation and two scenarios where private car, taxi and bus trips are replaced by automated taxis and automated taxis and shared taxis respectively. The fleet size of automated (shared) taxis is set at $4.8\%$ of the baseline vehicle fleet. In these scenarios about 50-70 \% of trips are serviced by the automated (shared) taxis which increases the vehicle utilization from $50 ~ \textnormal{mins}$ to $12.87\text{h}$, on average, per day. The authors project a $55 \%$ cost decrease, highly increased transportation accessibility in the city and carbon emission reductions of almost $40\%$. The simulation does not consider  changes on traffic density parameters resulting from self-driving vehicles. Furthermore, the demand is static and used preset parameters. Finally, the fleet control (rebalancing and dispatching) for the (shared) automated taxis is implemented based on heuristics and a local gradient-based optimization method, which was not compared to existing benchmark operational policies.

\citet{boesch2016autonomous} investigate a scenario of the greater Zurich region in Switzerland. They use a demand pattern for private vehicles generated with MATSim, which consists of 1.3 million private vehicle users and conclude that $30 \%$ of the substituted fleet can serve almost $100 \%$ of the substituted
requests with a wait time of less than $10 ~ \textnormal{mins}$.
%$1.3$ million  private vehicle users out of a total of
%$2.1$ million agents generate $3.6$ million trips. This demand profile generated
%with the co-evolutionary algorithm inherent to MATSim is then post-processed
%in a static simulation where $1-10 \%$ of the car trips are served by $10-100 \%$
%of the total number of substituted users. The authors conclude that approximately
A major limitation is the assumption that if wait time is surpassed, the request is dropped.  Furthermore, no rebalancing or dispatching is taking place; and a distribution of vehicles using the Eucledian distance in combination with a scaling factor is used instead of a network-based simulation.  
% furthermore network routing
% is not considered, travel times are based on Euclidean distance and a scaling factor.
%  The demand profile is static and does not vary depending on service times,
%  congestion rates and performance of the modes. 
%\hl{TODO I have added many ways in whic we improve the state of %the art compared to this work but they are commente. Please add %ome more information on what makes our approach novel in %comparison to this work, possibly uncommend some of the %information.}

In contrast to the Zurich study presented above, a case study for Berlin presented in \citep{bischoff2016simulation} simulates a fleet of AVs reacting to spatially and temporally distributed requests. It considers a city-wide replacement of private vehicles with automated taxis, that are dispatched using a heuristic algorithm, which will be tested in the present work.
 %The dispatching of the car works according to a policy that distinguishes
 %between oversupply (more available vehicles than open requests) and undersupply
 %and matches the closest vehicle to an appearing request, the next available
 %vehicle to the closest request respectively. Using this strategy called single
 %heuristic dispatcher in our work, the authors are able to serve $4.7$ million
 %requests generated by $1.1$ million car users with a fleet of $100,000$ autonomous
 %vehicles. The recorded average wait time for this case is about $2.5 \%$ minutes
 %and the $95\%$ quantile approximately $8.5$ minutes.
 The study concludes that 1.1 million former car users can be served by a fleet  of 100,000 shared automated vehicles.
% The resulting sharing factor
% is approximately $10$ to $12$.
The study is one of the first large-scale dynamic simulations of a shared automated taxi system, however it does not consider  different rebalancing and dispatching strategies. The work is extended in \cite{review3} to assess the impact of AV fleets on congestion in mixed traffic conditions; according to the results, the capacity gains by automation that are expected in today's literature are likely to exceed the losses caused by additional vehicle distance.

To summarize, there are numerous studies on optimal dispatching algorithms for automated vehicles, and there are also many attempts to simulate such fleets on a larger scale. However, most of the studies are based on certain common assumptions and simplifications:

\begin{itemize}
    \item Often, only one operational policy is evaluated; a rigorous comparison of different strategies on the same simulation setup has not yet been conducted.
    \item Vehicle movements are not simulated explicitly on a road network, but in a flow-based way. Often this is done on complete graphs with relatively small numbers of vertices.
    \item Constant levels of congestion are used, independent of traffic resulting from background traffic and/or movement of the automated vehicles themselves.
    \item Operational constraints, such as parking or vehicle recharging activities, are not simulated and/or taken into account.
    \item Computer-generated demand profiles are used, which do not accurately capture the complex time-varying and spatially biased distribution of origins and destinations in a real transport system.
    \item Usually, a static demand assumption is made, where the travellers do not dynamically react to the service offered by their travel decisions. Often, the demand describes some kind of optimal situation, like the conversion of all existing car trips.
    \item Simplified fallback strategies are used if wait times for specific customers get too long. In those cases, fleets are not required to serve every request.
    \item Small sample sizes of the simulation scenario are used to manage  computational demand. The effects of those scalings are usually unclear.
\end{itemize}

Considering how fast the research community is developing models, the authors expect these shortcomings to be alleviated in pending publications. The present paper contributes to the ongoing research as follows: as in past efforts, we present a large-scale agent-based simulation study for the management of automated vehicle fleets. In contrast to previous work, we rigorously compare multiple operational policies and simulate them on a detailed road network.  Our demand and our fleet sizes are on a realistic scale; we do not simulate only a sample of requests. We do so with realistic demand patterns derived from real-world travel data.  Our work is available in the open source and all compared algorithms are available online. They can be tested with custom-made simulation scenarios on the AMoDeus platform \citep{amodeusBase}.






%======================== END of new part ============================================

%Let's go through this again. We need:

%\begin{itemize}
%\item Literature on rebalancing in general (Claudio bike rebalancing, but there is a lot for car-sharing available), just indicate importance not go into detal on algorithms
%\item Simulation studies on AVs (we have that already! commented out right now), start with the conceptual ones (Spieser) and mention we take algos from there
%\item Literature MATSim
%\end{itemize}

%It goes as follows: First literature on rebalancing is introduced, then the simulation studies on AVs
%are presented with the remark that they do NOT consider rebalancing. We end with Kockelman and Bösch,
%who use MATSim but just as a preparational step and lead to full MATSIm simulations, first Michal & Joschka,
%then ABMTRANS, which is used here.

%This way we have introduced everything we're using here.



%While

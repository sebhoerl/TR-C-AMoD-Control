\section{Introduction}

With rapid technological developments in recent years has led to the point where automated vehicles are being tested in pilot projects around the world \citep{ackerman2017hail}. They promise to increase speeds, use road capacities more efficiently \citep{Tientrakool2011,Friedrich2015} and would give mobility access to formerly excluded user groups \citep{Truong2017}. On the other side, an increase of vehicle miles travelled (VMT) is expected due to empty rides \citep{Litman2014}, and the general increase of users could potentially congest the urban environment further \citep{Meyer2017}. Hence, net effects on the transportation system, environment and society are unclear. Simulations, such as the one presented in this work, can help to better understand the impact of future developments in vehicle automation and, consequently, how those affect technology diffusion scenarios discussed today \citep{NIEUWENHUIJSEN2018300}.

A number of recent studies have debated the feasibility of an automated mobility-on-demand (AMoD) system (see Section  \ref{subs:literatureResearch}). An AMoD system would allow travelers to be picked up at any time and location, to be transported to their desired destination by an automated vehicle (AV). Without owning a car, they would purchase mobility as a service. For the customer, this would offer the convenience of an individual taxi service for a fraction of today's cost. It is projected that the costs of using the service on a daily basis could be competitive with privately owned cars and even
public transit \citep{Bosch2016a}.

The success of an AV operator would depend on the pricing of the service,
as well as potential wait and travel times that can be offered. While high prices
may restrict the user group drastically, long wait times might have the same effect
if they make travelling less predictable. Both quantities are inherently
linked to two different measures of efficiency: first, capital and second, operational expenditure of the mobility scheme. A small fleet size
will have smaller capital expenditures, but inevitably increases waiting times. Operationally, an efficient use of vehicles restricts the number of empty miles driven, but also increases wait times, as vehicles without a passenger on board cannot be relocated to high demand locations.


%\sout{: If wait times have to be minimized, vehicles have to be at all times present where the demand is expected. This makes it necessary to relocate them without a passenger on-board, which directly translates to costs for
%the operator. Furthermore both quantities are also linked to the vehicle fleet size, which heavily influences both cost and wait times.}

In this study we contribute to research on AMoD systems. We
(a) present a simulation scenario of a fleet of an automated taxi fleet for Zurich, Switzerland,
based on the MATSim framework \citep{Horni2015}. We then (b) test and compare four different operational policies (from literature) for different fleet sizes on that infrastructure, and (c) analyze the results as they relate to 
 customer experience. The simulations performed do not use any down-sampling of the actual number of requests expected in reality and are performed on a detailed road network. They are based on realistic temporal and spatial demand patterns derived from real-world travel data.

% Commented out to reduce words. Do we need this? /sh

%The remainder is structured as follows: First, an overview of related search is
%given, then the simulaton scenario and environment are introduced, as well as the
%proposed fleet control algorithms. Thereafter, simulaton results are presented and
%analysed, followed by a discussion of our findings.


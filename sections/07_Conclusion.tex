\section{Conclusion}
\label{sec:Conclusion}

This study shows that, even under ideal conditions (highest possible demand, free-flow speeds), an AMoD service could not be operated at a price per kilometer lower than 0.4 CHF/km. However, under the assumed conditions, the service could be offered for 0.45 CHF/km to 0.7 CHF/km for peak wait times of 5 to 3.5 minutes for 90\% of the trips. For that, fleet sizes of about 7,000 to 14,000 automated taxis would be necessary, while less or more vehicles would either lead to extremely high wait times, or high prices.


In any case, price ranges described cannot compete with conventional public transport or private cars in the short term if only monetary costs are compared. Hence, future research is needed that explores the trade-off between monetary travel costs, value of time and customer acceptance in general. Also, we find that taking investment and maintenance costs of private car ownership into account, the AMoD services become highly competitive, which can be interpreted as a strong argument for the transition of classic car ownership to the sharing economy.


The results show that the utilization of intelligent demand forecasts and dispatching and rebalancing algorithms is crucial and effective to gain a competitive advantage when operating an AMoD fleet. We show that similar improvements in wait times can be achieved by either increasing the fleet size, which is probably costly, or simply applying a more intelligent operating policy. Even with the simple approaches presented in the paper, considerable differences between strategies can be observed and much more intelligent and capable algorithms are expected to evolve.


From our experiments, it is clear that the choice of fleet operational policy is a crucial part of any simulation assessment that has been, and will be, made around fleets of (autonomous) taxi vehicles. In fact, very simplistic approaches, which have been common in system-wide simulation assessments to date, may strongly underestimate potential benefits of such services for the transport system. Therefore, we strongly advise close examination of the respective operational policies when interpreting existing and future simulation assessments of such systems.
We evaluate performance of four different operational policies to control of an automated mobility-on-demand system with sequential vehicle-sharing is evaluated in one public, open-source-accessible,  agent-based, high fidelity simulation environment. Detailed network dynamics on a road level of precision are taken into account. The case study is conducted in a simulation scenario of Zurich city. The results indicate that automated vehicles’ shared mobility systems can provide approximately six times higher occupancy rates than conventional private cars, but that their costs - in Switzerland - are considerably higher than those of subsidized public transport, or private cars, in the short term. However, these services are predicted to be considerably cheaper than the full costs of owning and using a private car, which makes the long-term adoption of automated taxi services likely. Simulations demonstrate that the choice of the fleet operational policy determining customer-vehicle assignment and repositioning of empty vehicles (rebalancing) heavily influences system performance, e.g., wait times and cost. This aspect must thus be regarded as an important factor in the interpretation of existing and future simulation studies.